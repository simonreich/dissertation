I supervised numerous Bachelor and Master students, who contributed work to \charef{cha:perception}.

\paragraph{Georg Jahn} finished two Master's Theses in our lab, one for computer science and one for physics, in 2014.
He assembled the FlyPi robot, included a Kalman- and PID-Controller, and added a front-facing camera to the quadcopter.
He invented a marker system used for navigation, which is more error robust than most current systems and can be read by a fast moving robot.
The quadcopter navigates using the marker system.
Student work: 80\%, Supervisor work: 20\%.
The supervisor contributed as: theses supervision, project guidance, and detailed knowledge about data filtering, and hardware design.

\paragraph{Daniel Kalin} wrote a Bachelor's Thesis in computer science in 2015.
He assembled the first WheelPi robot and adapted the FlyPi robot-specific code to a slow-moving ground based robot.
Additionally, he added a SLAM algorithm based on an ultrasonic sonar.
Student work: 85\%, Supervisor work: 15\%.
The supervisor contributed as: thesis supervision, project guidance, code review, detailed knowledge about electronic circuits, detailed knowledge about data filtering, and detailed knowledge about SLAM on embedded hardware with limited memory.

\paragraph{Martin Heinemann} finished his Bachelor's Thesis in computer science in 2016.
He assembled the second WheelPi robot and enhanced some of the wiring, \eg removed an Arduino Board and moved its workload onto the Raspberry Pi computer.
Additionally, he worked on a stitching algorithm, which combines two overlapping images into one large image.
This algorithm is used to fuse multiple maps into one.
Both WheelPi robots now can record maps and share knowledge of obstacles.
Student work: 80\%, Supervisor work: 20\%.
The supervisor contributed as: thesis supervision, project guidance, code review, detailed knowledge about electronic circuits, and computer vision methods on embedded hardware with limited computational power.

\paragraph{Caroline Campbell} wrote her Bachelor's Thesis in physics in 2016.
The goal of her work was to learn the three static parameters of a PID control system using a shallow neural net.
Furthermore, she introduced a simulation environment for the FlyPi robot.
Student work: 90\%, Supervisor work: 10\%.
The supervisor contributed as: thesis supervision, project guidance, and detailed knowledge about learning algorithms.

\paragraph{Jan Lukas Bosse and Johannes Otto} worked in the lab during an internship over the course of one semester.
First, they measured the performance of the robot's \gls{ac:imu}.
Second, this data was used to estimate the robot's position via a Kalman Filter.
Student work: 80\%, Supervisor work: 20\%.
The supervisor contributed as: supervisor, project guidance, detailed knowledge about electronic circuits, and Kalman Filter design.

\paragraph{Lars Berscheid} worked on a Master's Thesis in physics in 2016.
He introduced an omnidirectional camera setup, computed features on the image stream, and from those he calculated the optical flow.
This is used to infer the robots offset from one frame to the next and is called \gls{ac:vo}.
Student work: 90\%, Supervisor work: 10\%.
The supervisor contributed as: thesis supervision, project guidance, detailed knowledge about electronic circuits, and computer vision methods on embedded hardware with limited computational power.

\paragraph{Damian Bast} worked on a Bachelor's Thesis in computer science in 2017.
In his thesis he built an algorithm that learns relationships between the robot's actuators and its sensors.
Thus, a forward-model of the robot is generated.
Student work: 90\%, Supervisor work: 10\%.
The supervisor contributed as: thesis supervision, project guidance, and detailed knowledge about learning algorithms.

\paragraph{Maurice Seer} finished his Master's Thesis in physics in 2018.
He added a laser pointer to the quadcopter, which is able to point at a specific point in space while the robot is flying.
Moreover, he significantly refined the \gls{ac:vo} algorithm and introduced a benchmark, which allows comparison to other methods.
He estimates the depth of the features, which are used to compute the optical flow.
Additionally, he and Lars Berscheid integrated the existing system into \gls{ac:ros}.
This allowed for a modular design structure.
Student work: 85\%, Supervisor work: 15\%.
The supervisor contributed as: thesis supervision, project guidance, detailed knowledge about electronic circuits, detailed knowledge about \gls{ac:ros} nodes, and computer vision methods on embedded hardware with limited computational power.

\paragraph{Philipp D\"onges} worked on his Master's Thesis in physics in 2018.
He added the \gls{ac:vo} setup and a motor odometer to the WheelPi robots.
This allowed to benchmark the \gls{ac:vo} algorithm not only in simulation, but also in a real world experiment.
Additionally, he significantly improved the feature's depth estimate.
Student work: 85\%, Supervisor work: 15\%.
The supervisor contributed as: thesis supervision, project guidance, detailed knowledge about electronic circuits, detailed knowledge about \gls{ac:ros} nodes, and computer vision methods on embedded hardware with limited computational power.

\paragraph{Erik Schultheis} worked on his Master's Thesis in physics in 2018.
Similarly to Caroline Campbell, in his thesis he learned the three static parameters of the PID controller using a shallow neural net.
He realized that this learning method is too limited as it does not take long term effects into account.
Therefore, it was extended to an actor-critic system with deep deterministic policy gradient.
He showed that in this system the policy-net can be replaced by a much simpler system, \ie a PID controller, which in turn converges significantly faster.
Student work: 90\%, Supervisor work: 10\%.
The supervisor contributed as: thesis supervision, project guidance, and detailed knowledge about machine learning.

\paragraph{Kevin Vorwerk} wrote his Bachelor's Thesis in computer science in 2018.
In his work he implemented a SLAM algorithm based on the \gls{ac:vo} setup.
Student work: 85\%, Supervisor work: 15\%.
The supervisor contributed as: thesis supervision, project guidance, detailed knowledge about SLAM algorithms, and detailed knowledge about computer vision methods on embedded hardware with limited computational power.
