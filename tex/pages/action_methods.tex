\section{Methods}
\label{sec:action_methods}

\begin{figure}[]
  \centering
  % Define block styles
\tikzstyle{block} = [draw, rectangle, fill=white, minimum height=3.5cm, minimum width=6.0cm, text width=4.5cm, text centered, rounded corners=true]
\tikzstyle{blockBox} = [draw, inner sep=0.2cm, dashed, rectangle, minimum width=3.5cm, text centered, rounded corners=true]
\tikzstyle{blockText} = [text centered, minimum height=1.5cm,text width=5.0cm]
\tikzstyle{arrow} = [draw, -latex]

\definecolor{red1}{RGB}{160,0,0}
\definecolor{green1}{RGB}{0,160,0}
\definecolor{blue1}{RGB}{0,0,160}

% Define images
\pgfdeclareimage[height=3.0cm]{scene}{./figures/sensor/methods_flowchart_scene.jpg}
\pgfdeclareimage[height=3.0cm]{robot}{./figures/sensor/methods_flowchart_robot.jpg}
	      
\begin{tikzpicture}[node distance = 8.0cm, auto]
  % Scene
  \node[block] (scene) {\pgfuseimage{scene}};

  % Hardware setup
  \node [block, right of=scene] (hardwaresetup) {\pgfuseimage{robot}};

	% Filter
  \node [block, below of=scene, node distance=6.5cm] (filter) {Removing noise and outliers};
  
  % VO
  \node [block, right of=filter] (vo) {Estimating pose, position, and environment};
  

  % Text
  \node [blockText, above of=scene, node distance=2.5cm] (scene_t) {\textbf{a) Scene}};
  \node [blockText, above of=hardwaresetup, node distance=2.5cm] (hardwaresetup_t) {\textbf{b) Hardware setup}\\\secref{ssec:perception_methods_hardwaresetup}};
  \node [blockText, above of=filter, node distance=2.5cm] (filter_t) {\textbf{c) Preprocessing}\\\secref{ssec:perception_methods_noiseandoutlierdetection}};
  \node [blockText, above of=vo, node distance=2.5cm] (vo_t) {\textbf{d) Visual Odometry}\\\secref{ssec:perception_methods_visualodometryalgorithm}};


  % Fit
  \node[fit=(scene)(scene_t), blockBox] (scene_f) {};
  \node[fit=(hardwaresetup)(hardwaresetup_t), blockBox] (hardwaresetup_f) {};
  \node[fit=(filter)(filter_t), blockBox] (filter_f) {};
  \node[fit=(vo)(vo_t), blockBox] (vo_f) {};


  % Arrows
	\path [arrow] (scene_f.east) to (hardwaresetup_f.west);
	\path [arrow] (hardwaresetup_f.south) -| ++(0,-0.55cm) -| (filter_f.north);
	\path [arrow] (filter_f.east) to (vo_f.west);
\end{tikzpicture}

  \caption{Flowchart of the methods in this chapter and how they relate. Details are explained in \secref{sec:action_methods}.}
  \label{fig:actions_methods_flowchart}
\end{figure}

In this section, approaches to planning are introduced.
An overview is given in \figref{fig:actions_methods_flowchart}: First, a) the scene is recorded as RGB-D image by an Asus Xtion Pro camera and a Nikon high resolution \gls{ac:dslr}.
Each depth frame is b) preprocessed by computer vision algorithms.
They are segmented via the Locally Convex Connected Patches (LCCP) algorithm~\cite{stein2014object} into clusters, where each cluster correlates to one object candidate.
For object recognition~\cite{schoeler2014fast} is used; objects are tracked via~\cite{paponkulviciusaksoy2013}. 
All parts of the algorithms are integrated as modular \gls{ac:ros} nodes~\cite{quigleyconleygerkey2009}.
Next, the objects are categorized based on c) an action ontology, \secref{ssec:action_methods_actioncategories}.
This categorization is used throughout the rest of this thesis.
Based on this information, a graph structure is retrieved in d) \acrlong{ac:sec} (\secref{ssec:action_methods_semanticeventchains}).
As first contribution, the graph is enriched with pose information by e) a novel 3d reasoning algorithm (\secref{ssec:action_methods_enrichedsemanticeventchains}).
Afterwards, it is shown in f) that each graph can be reduced to only three different subgraphs (\secref{ssec:action_methods_structuralinformation}).
This in turn is used to compute g) a scene's affordance (\secref{ssec:action_methods_affordanceofsemanticeventchains}).
Here, for example the question is asked: ``Given a tomato, a knife, and a cutting board, what can one do with these objects?''
In h), it is shown how the approach can be extended for complex action planning (\secref{ssec:action_methods_usingaffordanceforplanning}).
As an example it is analyzed what actions a robot needs to perform when given the command ``Make me a sandwich.''
